\documentclass{article}
\bibliographystyle{plain}
\usepackage{amssymb,amsmath}
\begin{document}
\title{Preserving Antebellum Print Culture}
\author{Jason Peak}
\maketitle


\def\projectname{Poe's Magazine World}
\def\APC{Antebellum Print Culture}
\def\bwj{\emph{The Broadway Journal}}
\def\slm{\emph{Southern Literary Messenger}}
\def\bgm{\emph{Burton's Gentleman's Magazine}}
\def\gm{\emph{Graham's Magazine}}

\section{Project Scope}
\projectname, a digital humanities project centered around the literary culture of the 1840s in America, seeks to accumulate and present digital artifacts that re-contextualize this period in history for modern readers and researchers. The project team is composed of a cross-disciplinary mix of scholars and students committed to establishing a research archive for the equally diverse audience of scholarly researchers, students and the general public. At its broadest, the scope of the project includes all American Antebellum authors and would take shape as part of a consortium of linked archives. As a first step, however, the project will focus on Poe's involvement with four periodicals, the \bwj, \slm, \bgm and \gm. The main tasks involved are summarized as follows:

\begin{description}
  \item[Scan]{Each of the 4,000 pages contained in the corpus in scope will be scanned to high resolution TIFF}
  \item[Extract]{OCR and minimal human processing will result in level 2 TEI\cite{bpgtei} markup for each page}
  \item[Preserve]{These artifacts must be preserved in a trusted system}
\end{description}

As a result of these three tasks, the initial phase will establish a trusted digital repository (TDR) suitable as a research resource. Facilities for research will include, at a minimum, document retrieval and full-text search. With these basic foundation blocks firmly laid, other services can be built that further enhance the value of this resource for research including scholarly annotation, use of RDF for making semantic inferences, and a range of client web services that expose the archive to other repositories and applications.

\section{Requirements}

\begin{description}
  \item[Archival Preservation] The APC project requires trusted archival preservation of its primary artifacts and derivative scholarship and interactions.
  \item[Exposure] Artifacts and their metadata must be exposed to client applications including search engines, browsers, and visualization applications
  \item[Granular markup] Text documents will be encoded in the TEI at the various levels enumerated in \cite{bpgtei} 
\end{description}



\bibliography{bib}

\end{document}
