Software packages and cloud services exist that provide some or all of the functions required of a digital repository. Off the shelf, a number of candidates present themselves. Fedora, Archivematica, DAITSS and ContentDM are the most prominent. 
\paragraph{ContentDM}
Because ContentDM is a closed-source product, we will not consider it a viable option. 
\paragraph{FEDORA}The Flexible Extensible Digital Object Repository Architecture is a back-end system that manages ingest, content, and policies. Fedora offers flexible data models that let repositories represent arbitrarily complex objects consisting of a range of media types. At the core of this model is extensive use of the METS metadata standard format to encapsulate structural, administrative, representational, and descriptive metadata\cite{fedora-meta-phil}.
\subparagraph{Islandora} is a Drupal-based front-end for Fedora. Drupal is a popular, widely-used, community-developed, open source web platform used for the rapid development of web applications. While there is an online sandbox\cite{islandora-sandbox} for evaluation purposes, any local investigation into the use of Fedora should include an Islandora front-end.
\paragraph{DAITSS}Dark Archive in the Sunshine State is designed for long-term preservation\cite{daitss} and, as its name suggests, places great emphasis on the integrity of the dark archive. Having recently released their 2.0 version touting RESTful web services and multiple tenancy, DAITSS should be included on the short list for further investigation. 
\paragraph{Archivematica} is new, and is poised for its 1.0 release. Archivematica claim OAIS compliance\cite{archivematica-main} and ingest-to-access workflow based on a microservices architecture.