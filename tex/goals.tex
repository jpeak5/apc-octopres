\section{Core Goals and Requirements}
For the editions of the magazines that Poe edited, the page count is about 4,000. The primary project goal is the establishment of a digital repository containing the text and scanned image of each of these pages. In addition to making each page image available,the text should be fully searchable. Providing access to page images and text search tools implies an access-controlled, web-based interface. Maintaining the integrity of complex, mixed-media, multi-page document records requires sturdy archival frameworks and flexible metadata standards. 


\subsection{Additional Opportunities}
With core foundation blocks laid, the team will build up enhanced access services, including annotation, semantic search, and the infrastructure to expose the repository as linked open data. 
\begin{description}
  \item[Annotation] consists of moderated user annotations to the elements of the repository. These annotations may become part of the archival record.   
  \item[Search] entails semantic markup. The team will identify or develop an appropriate ontology with which to describe semantic relationships amongst repository elements at the sub-document level. Initial candidates for such markup include people, places, corporate bodies captured with EAC-CPF. 
  \item[Linked Open Data] Client webservices will be developed  to allow others to reuse the archive as a dataset for their own purposes. 
\end{description}
Access to the contents of the repository is a key driver of this effort. While our core goals will lead us to a stable and sustainable repository, opening the contents of that repository to outside research is where the project gains traction, value, and the sustainable funding required for permanence. In the following sections, we will explore the issues associated with both the core and stretch goals and recommend tools, standards and workflows for achieving them.