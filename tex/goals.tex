\subsection{Core}
Immediate project goals include digitizing each page of every periodical volume of which Poe took part. Considering only the years in which he was actively employed by these editorials, the corpus contains just less than 4,000 pages. The primary deliverable for the next stage of this project will be a permanent, web-accessible repository consisting of a scan and fully-searchable text for each page of each publication. The tasks, explored below, required to achieve core goals include scanning, text extraction, curation, and preservation. User-facing facilities for research will include, at a minimum, document retrieval and full-text search. 

\subsubsection{Stretch}
With core foundation blocks firmly laid, other services may be built that further enhance the research value of this resource including scholarly annotation, semantic curation, and a range of public APIs that expose the archive to other repositories and applications as Linked Open Data. Annotation will consist of moderated scholarly annotations made by local and distributed domain experts.  Semantic curation entails the identification and markup of the occurence of significant entities within the text, including people, places, corporate bodies. Client webservices will be developed  to allow others to reuse the archived dataset for their own purposes. In light of the current trend \needcite in archives towards Linked Open Data (LOD) and the aspirations of the project team, it is reasonable to consider the APC project well positioned to be one of the next notable contributors to this burgeoning community of archives.