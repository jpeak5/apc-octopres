The term \emph{Trusted Digital Repository} is loaded with implications, most importantly, that some digital repositories are not trusted. Dow prefers to use \emph{Trustworthy} Digital Repository, and gives its meaning as \emph{an overall commitment to the stewardship of digital materials}\cite{dow_elizabeth_2009}. As a purely digital project, the long-term success of \projectname is bound to making this commitment. While there is no official definition\cite{dow_elizabeth_2009} of TDR, there is general agreement of its components and better, standards for determining \emph{trustworthiness}.
\subsection{TDR Components}
\subsubsection{OAIS}
There are several models used in building trusted digital repositories, but by far the most common is based on the Open Archives Information System (OAIS) model, ISO 14721:2003. OAIS is a high-level model, and the details of implementation are determined locally. The model defines six functions that a conforming repository must implement.
\begin{enumerate}
  \item{Ingest}
  \item{Archival}
  \item{Data Management}
  \item{Administration}
  \item{Access}
  \item{Preservation Planning}
\end{enumerate}

\subsection{TDR Requirements}
\begin{quote}A sustainable, trustworthy, well-supported, and well-managed digital repository needs hardware, software, policies, processes, services, and people to assure long-term retention and, perhaps, access to its content and metadata.\cite{dow_elizabeth_2009} \end{quote}

\subsection{TDR Implementations}
Some examples of such systems include ContentDM, Fedora, DSpace. Some of these are available as hosted or self-hosted solutions. In either case, standards exist \needcite [DRAMBORA, TRAC] by which institutions can measure the trustworthiness of the repository.