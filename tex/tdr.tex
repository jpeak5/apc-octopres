The term \emph{Trustworthy Digital Repository} is loaded with implications and requirements. Dow gives one meaning of the term \emph{Trustworthy} Digital Repository as an \emph{overall commitment to the stewardship of digital materials}\cite{dow_elizabeth_2009}. Such repositories are composed of standards-based frameworks built to operate on standards-based records and to enforce standards-based archival policies; most of these are expressed as XML. Replacing the current infrastructure with such a system increases complexity and potential cost but offers a commensureate increase in long-term stability, sustainability and interoperability. 
\subsection{TDR Requirements}
\begin{quote}A sustainable, trustworthy, well-supported, and well-managed digital repository needs hardware, software, policies, processes, services, and people to assure long-term retention and, perhaps, access to its content and metadata.\cite{dow_elizabeth_2009} \end{quote}

\subsection{TDR Implementations}
Some examples of such systems include ContentDM, Fedora, DSpace. Some of these are available as hosted or self-hosted solutions. In either case, standards exist \needcite [DRAMBORA, TRAC] by which institutions can measure the trustworthiness of the repository.