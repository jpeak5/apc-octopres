The term \emph{Trusted Digital Repository} is loaded with implications, most importantly, that some digital repositories are not trusted. Dow prefers to use \emph{Trustworthy} Digital Repository, and gives its meaning as \emph{an overall commitment to the stewardship of digital materials}\cite{dow_elizabeth_2009}. As a purely digital project, the long-term success of \projectname is bound to making this commitment. While there is no official definition\cite{dow_elizabeth_2009} of TDR, there is general agreement of its components and better, standards for determining \emph{trustworthiness}.
\subsection{TDR Components}
\subsubsection{OAIS}
There are several models used in building trusted digital repositories, but by far the most common is based on the Open Archives Information System (OAIS) model, ISO 14721:2003. OAIS is a high-level model, and the details of implementation are determined locally. Using the model helps organizations plan sustainable repositories\cite{harvey} The model defines six functions that a conforming repository must implement.
\begin{enumerate}
  \item{Ingest}
  \item{Archival}
  \item{Data Management}
  \item{Administration}
  \item{Access}
  \item{Preservation Planning}
\end{enumerate}

\subsection{Building a TDR}


\subsection{TDR Requirements}
\begin{quote}A sustainable, trustworthy, well-supported, and well-managed digital repository needs hardware, software, policies, processes, services, and people to assure long-term retention and, perhaps, access to its content and metadata.\cite{dow_elizabeth_2009} \end{quote}
Software packages and cloud services exist that provide some or all of the functions required for a TDR. The computational machinery required is not particularly challenging, rather, it is the soft machinery of people, policies and planning that offer the greatest challenge.

Off the shelf, a number of candidates present themselves. Fedora, DSpace, and ConetntDM are the most prominent, and of these, only ContentDM is proprietary, making it a somewhat less desirealble choice.

The Flexible Extensible Digital Object Repository Architecture (Fedora) is a backend system that manages content, policies. 