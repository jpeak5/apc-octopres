Currently, the project uses the popular Omeka publishing platform as repository, content ingest and administration interface, and dissemination portal. The application relies on a relational database to maintain application state and configuration, to control user access, for persistent records and metadata storage, and for RDF-like inter-record relationship data. 

Besides offering a one-stop solution to the needs of a small repository, Omeka is freely available and open source from top to bottom. It is written in PHP and based on the industry standard Zend PHP framework. It can be fitted to a number of databases, including PostgreSQL and MySQL, which are also free and open source. Rounding out the Omeka architecture is the apache webserver running on linux. Using open source software allows the team great flexibility in customizing the software, and independence in resolving technical issues. As the project goals outgrow the capabilities of Omeka, finding a similarly open replacement will be in keeping with archival best practices \needcite


