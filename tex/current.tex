Currently, the project uses the popular Omeka publishing platform as repository, content ingest and administration interface, and dissemination portal. The application relies on a relational database to maintain application state and configuration, to control user access, for persistent records and metadata storage, and for RDF-like inter-record relationship data. 

Besides offering a one-stop solution to the needs of a small repository, Omeka is freely available and open source from top to bottom. It is written in PHP and based on the industry standard Zend PHP framework. It can be fitted to a number of databases, including PostgreSQL and MySQL, which are also free and open source. Rounding out the Omeka architecture is the Apache webserver running on linux. Using open source software allows the team great flexibility in customizing the software, and independence in resolving technical issues. As the project goals outgrow the capabilities of Omeka, finding a similarly open replacement will be in keeping with archival best practices \needcite

\subsection{Constraints}
While Omeka has served the project well, achieving the core project goals within its current framework presents significant challenges and forces some awkward compromises  with respect to the internal representation of data. Specifically, Omeka is not well-suited to multi-page records or sub-document references. 

Omeka operates on \emph{items}. Each item is defined by its Dublin Core metadata, and files, such as images, can be attached to item records. This model fits a collection of paintings, photographs, or museum objects very well; these can be gathered into an Omeka \emph{collection} and presented as an \emph{exhibit}. Approximating a magazine, consisting of succeeding pages would require distinct items for each page, using the page number as the title, perhaps, attaching the appropriate page scan to the appropriate item, and then assembling all of the pages into a collection or exhibit. Assuming that the page order can be defined for the exhibit, the structure that it represents is an attribute of the exhibit, not of the \emph{item} itself. 

There are well-known XML dialects that are designed to describe such multi-page documents, but put simply, Omeka does not speak XML. Certainly, this team has written custom plugins to give Omeka the ability to parse XML, specifically TEI, documents, but because Omeka was not designed to operate on XML data, this approach is not sustainable in terms of long-term preservation or the extended goals this project aims to achieve. Trustworthy digital repositories are composed of standards-based frameworks built to operate on standards-based records and to enforce standards-based archival policies; most of these are expressed as XML. Replacing the current infrastructure with such a system increases complexity and potential cost but offers a commensureate increase in long-term stability, sustainability and interoperability. 