Workflow is loosely defined here. Inevitably, refinements will be made as issues are discovered and resolved. The acquisition process, from capture to ingest, can be imagined as a pipeline starting from the physical original and ending with an archival Submission Information Package (SIP). Stages in this pipeline include: 
\begin{enumerate}
  \item{Scan}
  \item{OCR}
  \item{OCR correction}
  \item{TEI markup}
  \item{packaging}
\end{enumerate}
%This chain of functions can be represented as $SIP ::= TEI(text(scan(page)))$
While each stage will require human intervention and oversight, the OCR correction stage is expected to incur the most significant outlay of human resources. 
\subsection{Scan}
We intend to use facilities in the Hill Memorial Library for the bulk of the scans. This equipment is designed to scan a full page spread, two facing pages, at a time, but each page is saved as a discrete file. Scanning will be applied to each page in each issue of each periodical in the project scope.
\subsection{OCR to Text}
Full-text search requires that the text be converted from the printed page to machine readable format.


The scanning process will operate at the page level and will require manual operation. It will take as input physical original peridical pages and return as output a simgle high resolution TIFF image per page and text extracted through OCR. The products of this process are the basis of the archive.
%\begin{description}
%  \item[periodical]{physical original magazine}
%  \item[page]{leaf of a periodical}
%  \item[scan]{digitization of leaf to TIFF}
%  \item[TIFF]{image of leaf}
%  \item[OCR]{process by which text is extracted from TIFF}
%  \item[raw text]{output of OCR}
%  \item[OCR correction]{manual process by which machine interpretation errors are corrected by humans}
%  \item[TEI-1]{baseline format of text for archival storage}
%  \item[SIP]{METS-encoded record of a single periodical issue containing TIFF + TEI of each leaf contained in the original}
%\end{description}

