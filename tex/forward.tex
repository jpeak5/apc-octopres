Moving towards accomplishment of the core and stretch goals outlined previously, it is evident that the project should be ready to adopt new technologies into its existing information architecture. At the base, their must be a robust and trusted repository in which born-digital artifacts are deposited. The time and expense required to capture this corpus deserves no less, and we imagine that generations of future scholars and students will benefit from this effort, provided that the materials are well-preserved in accordance with the best practices currently known to the archives community. The term \emph{Trusted Digital Repository} is loaded with implications and requirements that we will endeavor to satisfy. Standards exist \needcite [DRAMBORA, TRAC] by which to measure and guide these efforts.

Further, in an increasingly interconnected and socially interlinked world, the digital archive assumes a certain burden to provide its content to patrons (researchers, students, the public) in increasingly compelling ways. One of the most promising technologies for achieving this end is embodied in the concept of a Semantic Web \needcite[TBL]. The semantic web represents an inductive step beyond the character of the web we've grown to know in the last several decades. In this new web, hyperlinks are not merely a means by which to refer to some other, arbitrary, web resource, rather, the reference itself is imbued with an additional dimension that describes the nature of the relationship between the thing and what it points to. \needcite[Barthes?] Employing such technology in an archives setting improves discoverability by expanding the role of the archive from simple document retrieval to something much richer.

The foregoing discussion asserts that access to the contents of the repository is a key driver of the effort. Indeed, our core goals will lead us to a stable and sustainable repository of the corpus, but opening the contents of that repository to outside research is where the project gains traction, value, and the sustainable funding required for permanence. In the following sections, we will explore the issues associated with both the core and stretch goals and recommend tools, standards and workflows for achieving them.